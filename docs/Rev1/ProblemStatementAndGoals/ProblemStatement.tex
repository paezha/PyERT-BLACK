\documentclass{article}

\usepackage{tabularx}
\usepackage{booktabs}
\usepackage[normalem]{ulem}

\title{Problem Statement and Goals\\\progname}

\author{\authname}

\date{April 5, 2023}

%% Comments

\usepackage{color}

\newif\ifcomments\commentstrue %displays comments
%\newif\ifcomments\commentsfalse %so that comments do not display

\ifcomments
\newcommand{\authornote}[3]{\textcolor{#1}{[#3 ---#2]}}
\newcommand{\todo}[1]{\textcolor{red}{[TODO: #1]}}
\else
\newcommand{\authornote}[3]{}
\newcommand{\todo}[1]{}
\fi

\newcommand{\wss}[1]{\authornote{blue}{SS}{#1}} 
\newcommand{\plt}[1]{\authornote{magenta}{TPLT}{#1}} %For explanation of the template
\newcommand{\an}[1]{\authornote{cyan}{Author}{#1}}

%% Common Parts

\newcommand{\progname}{Software Engineering} % PUT YOUR PROGRAM NAME HERE
\newcommand{\authname}{Team 17, Track a Trace
\\ Zabrain Ali
\\ Linqi Jiang
\\ Jasper Leung
\\ Mike Li 
\\ Mengtong Shi
\\ Hongzhao Tan
} % AUTHOR NAMES                  

\usepackage{hyperref}
    \hypersetup{colorlinks=true, linkcolor=blue, citecolor=blue, filecolor=blue,
                urlcolor=blue, unicode=false}
    \urlstyle{same}
                                


\begin{document}

\maketitle

\begin{table}[hp]
\caption{Revision History} \label{TblRevisionHistory}
\begin{tabularx}{\textwidth}{llX}
\toprule
\textbf{Date} & \textbf{Developer(s)} & \textbf{Change}\\
\midrule
Sept 23 & Jasper Leung & Modified Problem Statement, Goals, and Stretch Goals\\
Sept 23 & Zabrain Ali & Modified Goals, Stretch Goals, Team Number, and Group Name\\
Sept 23 & Hongzhao Tan & Modified Goals, Stretch Goals, Stakeholders, and Environment\\
Sept 23 & Mengtong Shi & Modified Problem, Inputs and Outputs, Stakeholders, and Environment\\
Sept 23 & Mike Li & Modified Problem Statement, Input, Output, Hardware, and Goals\\
Sept 23 & Linqi Jiang & Modified Outputs Problems Statement and Goals\\
April 3 & Mengtong Shi & Modified Environment\\
April 5 & Jasper Leung & Modified Document to match Revision 1\\
\bottomrule
\end{tabularx}
\end{table}

\section{Problem Statement}

%\wss{You should check your problem statement with the
%\href{https://github.com/smiths/capTemplate/blob/main/docs/Checklists/%ProbState-Checklist.pdf}
%{problem statement checklist}.}
%\wss{You can change the section headings, as long as you include the required information.}

\subsection{Problem}
GIS-based Episode Reconstruction Toolkit (GERT) is a set of tools for working with GPS data and is used to identify stop and travel episodes and  match them to a network. The toolbox currently uses an external software ArcGIS Pro, which requires a license to be accessed. This makes GERT less accessible for most potential users and very expensive to maintain. In theory, \textcolor{red}{\sout{the re-implemented toolbox} a program which recreates the main functions of the toolbox} should be fully open-source and independent so that it can accomplish the same tasks without needing to license ArcPro.
\subsection{Inputs and Outputs}
    \subsubsection{Input}
    \begin{itemize}
    \item Geographic data
    \item Mobility data
    \item GPS stamps
    \end{itemize}
    \subsubsection{Output}
    \begin{itemize}
    \item Analysis of geographic data
    \item Network and traffic analysis
    \item Map matching
    \item Possible research results according to user requirement
    \end{itemize}
%\wss{Characterize the problem in terms of ``high level'' inputs and outputs.  
%Use abstraction so that you can avoid details.}

\subsection{Stakeholders}

\begin{itemize}
    \item School of Earth Environment and Society of McMaster University
    \item Supervisor of the project Dr. Antonio Paez and the potential users of the toolbox
    \item Professor of the 4G06 capstone course Dr. Spencer Smith
    \item Developers of this project Zabrain Ali, Linqi Jiang, Jasper Leung, Mike Li,  Mengtong Shi, Hongzhao Tan
\end{itemize}

\subsection{Environment}
    \subsubsection{Software}
    The software should be compatible with Python\textcolor{red}{\sout{, R}} and any required libraries.
    
    \subsubsection{Hardware}
    The hardware that would be utilized includes personal computers for development, and computers from \textcolor{red}{the} General Science Building computer lab to access ArcGIS.

%\wss{Hardware and software}

\section{Goals}
\begin{itemize}
    \item \textcolor{red}{\sout{Re-implement the features in GERT that use ArcGIS Pro packages with open-source packages and libraries and remove any use of ArcGIS in the project.} Re-create the main functions of GERT with a separate program that uses open-source packages and libraries that don't use ArcGIS.}
    \begin{itemize}
        \item Using open-source packages and libraries instead of ArcGIS will make the \textcolor{red}{\sout{GERT tools} new program that recreates the functions of GERT} free to use.
    \end{itemize}
    \item Document the changes made to \textcolor{red}{\sout{GERT when replacing ArcGIS} the new program and how it differs from GERT}.
    \begin{itemize}
        \item Having detailed documentation for the \textcolor{red}{\sout{modified GERT} new program} will ensure that existing users of GERT will be able to transition from using \textcolor{red}{\sout{the ArcGIS version easily} GERT} and understand the changes made. It will also be helpful for new users.
    \end{itemize}
    \item Modify existing project structure to be more organized and readable.
    \begin{itemize}
        \item Most of the code for the current project is stored in a single file, making it hard to read. Organizing the code and potentially splitting up the code \textcolor{red}{of the new program} into separate modules would be helpful for any users trying to use the code.
    \end{itemize}
\end{itemize}
\section{Stretch Goals}
\begin{itemize}
    \item Implement GUI for the \textcolor{red}{\sout{re-implemented tool} the new program}.
    \begin{itemize}
        \item The current GERT is run using the command line. Implementing a GUI \textcolor{red}{for the new program} will make it more intuitive, and may motivate new users to use it.
    \end{itemize}
    \item Implement new features or improve existing features in GERT.
    \begin{itemize}
        \item If the existing features of GERT are all \textcolor{red}{\sout{converted to Python} present in the new program} and there is still time, we can spend the \textcolor{red}{\sout{remainder} remaining time} improving these features, or add new features based on Dr.Paez's feedback\textcolor{red}{, such as visualization}.
    \end{itemize}
    \item Create tutorials for how to use \textcolor{red}{\sout{modified GERT} the new program}.
    \begin{itemize}
        \item \textcolor{red}{\sout{There is the documentation for GERT within the code, but no videos/manuals on how to use it.} Adding videos/manuals for how to use the new program} would make the tool more accessible.
    \end{itemize}
    \item Improve run time by implementing more efficient sorting and searching algorithms.
    \begin{itemize}
        \item After successfully \textcolor{red}{\sout{reverse engineering the toolbox} re-implementing the main features of GERT in the new program}, we would like to improve the speed of the software by implementing new \textcolor{red}{\sout{potential}} algorithms to efficiently handle large geographic data sets.
    \end{itemize}
\end{itemize}

\end{document}