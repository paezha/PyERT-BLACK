\documentclass[12pt]{article}

\usepackage{amsmath, mathtools}
\usepackage{amsfonts}
\usepackage{amssymb}
\usepackage{graphicx}
\usepackage{colortbl}
\usepackage{xr}
\usepackage{hyperref}
\usepackage{longtable}
\usepackage{xfrac}
\usepackage{tabularx}
\usepackage{float}
\usepackage{siunitx}
\usepackage{booktabs}
\usepackage{caption}
\usepackage{pdflscape}
\usepackage{afterpage}

\usepackage[round]{natbib}

%\usepackage{refcheck}

\hypersetup{
    bookmarks=true,         % show bookmarks bar?
      colorlinks=true,       % false: boxed links; true: colored links
    linkcolor=red,          % color of internal links (change box color with linkbordercolor)
    citecolor=green,        % color of links to bibliography
    filecolor=magenta,      % color of file links
    urlcolor=cyan           % color of external links
}

%% Comments

\usepackage{color}

\newif\ifcomments\commentstrue %displays comments
%\newif\ifcomments\commentsfalse %so that comments do not display

\ifcomments
\newcommand{\authornote}[3]{\textcolor{#1}{[#3 ---#2]}}
\newcommand{\todo}[1]{\textcolor{red}{[TODO: #1]}}
\else
\newcommand{\authornote}[3]{}
\newcommand{\todo}[1]{}
\fi

\newcommand{\wss}[1]{\authornote{blue}{SS}{#1}} 
\newcommand{\plt}[1]{\authornote{magenta}{TPLT}{#1}} %For explanation of the template
\newcommand{\an}[1]{\authornote{cyan}{Author}{#1}}

%% Common Parts

\newcommand{\progname}{Software Engineering} % PUT YOUR PROGRAM NAME HERE
\newcommand{\authname}{Team 17, Track a Trace
\\ Zabrain Ali
\\ Linqi Jiang
\\ Jasper Leung
\\ Mike Li 
\\ Mengtong Shi
\\ Hongzhao Tan
} % AUTHOR NAMES                  

\usepackage{hyperref}
    \hypersetup{colorlinks=true, linkcolor=blue, citecolor=blue, filecolor=blue,
                urlcolor=blue, unicode=false}
    \urlstyle{same}
                                


% For easy change of table widths
\newcommand{\colZwidth}{1.0\textwidth}
\newcommand{\colAwidth}{0.13\textwidth}
\newcommand{\colBwidth}{0.82\textwidth}
\newcommand{\colCwidth}{0.1\textwidth}
\newcommand{\colDwidth}{0.05\textwidth}
\newcommand{\colEwidth}{0.8\textwidth}
\newcommand{\colFwidth}{0.17\textwidth}
\newcommand{\colGwidth}{0.5\textwidth}
\newcommand{\colHwidth}{0.28\textwidth}

% Used so that cross-references have a meaningful prefix
\newcounter{defnum} %Definition Number
\newcommand{\dthedefnum}{GD\thedefnum}
\newcommand{\dref}[1]{GD\ref{#1}}
\newcounter{datadefnum} %Datadefinition Number
\newcommand{\ddthedatadefnum}{DD\thedatadefnum}
\newcommand{\ddref}[1]{DD\ref{#1}}
\newcounter{theorynum} %Theory Number
\newcommand{\tthetheorynum}{T\thetheorynum}
\newcommand{\tref}[1]{T\ref{#1}}
\newcounter{tablenum} %Table Number
\newcommand{\tbthetablenum}{T\thetablenum}
\newcommand{\tbref}[1]{TB\ref{#1}}
\newcounter{assumpnum} %Assumption Number
\newcommand{\atheassumpnum}{P\theassumpnum}
\newcommand{\aref}[1]{A\ref{#1}}
\newcounter{goalnum} %Goal Number
\newcommand{\gthegoalnum}{P\thegoalnum}
\newcommand{\gsref}[1]{GS\ref{#1}}
\newcounter{instnum} %Instance Number
\newcommand{\itheinstnum}{IM\theinstnum}
\newcommand{\iref}[1]{IM\ref{#1}}
\newcounter{reqnum} %Requirement Number
\newcommand{\rthereqnum}{P\thereqnum}
\newcommand{\rref}[1]{R\ref{#1}}
\newcounter{lcnum} %Likely change number
\newcommand{\lthelcnum}{LC\thelcnum}
\newcommand{\lcref}[1]{LC\ref{#1}}

\newcommand{\famname}{FamName} % PUT YOUR PROGRAM NAME HERE

\usepackage{fullpage}

\begin{document}

\title{Program Family Title \plt{\famname should appear in the title}} 
\author{\authname}
\date{\today}

\maketitle

~\newpage

\pagenumbering{roman}

\plt{The CA template is related to the SRS template.  Many of the 
sections are in common.  The notes and advice for the SRS template are
not reproduced here. Please have a look at the SRS template for advice.}


\plt{This CA template is based on \citet{Smith2006}.  An example for a family of
  material models is given in \citet{SmithMcCutchanAndCarette2017}.  This
  example is for a physics based family.  Often the families will be based on
  generic numerical techniques, rather than physics.}

\plt{A good mindset for thinking about the families is often to think of the
  family as providing a library of services, as opposed to a single executable.
  The library of services can be used to build an application that uses a subset
  of the services, which is like providing the smaller library as a single
  family member.}

\section{Revision History}

\begin{tabularx}{\textwidth}{p{3cm}p{2cm}X}
\toprule {\bf Date} & {\bf Version} & {\bf Notes}\\
\midrule
Date 1 & 1.0 & Notes\\
Date 2 & 1.1 & Notes\\
\bottomrule
\end{tabularx}

~\newpage
	
\section{Reference Material}

This section records information for easy reference.

\subsection{Table of Units}

Throughout this document SI (Syst\`{e}me International d'Unit\'{e}s) is employed
as the unit system.  In addition to the basic units, several derived units are
used as described below.  For each unit, the symbol is given followed by a
description of the unit and the SI name.
~\newline

\renewcommand{\arraystretch}{1.2}
%\begin{table}[ht]
  \noindent \begin{tabular}{l l l} 
    \toprule		
    \textbf{symbol} & \textbf{unit} & \textbf{SI}\\
    \midrule 
    \si{\metre} & length & metre\\
    \si{\kilogram} & mass	& kilogram\\
    \si{\second} & time & second\\
    \si{\celsius} & temperature & centigrade\\
    \si{\joule} & energy & Joule\\
    \si{\watt} & power & Watt (W = \si{\joule\per\second})\\
    \bottomrule
  \end{tabular}
  %	\caption{Provide a caption}
%\end{table}

\plt{Only include the units that your CA actually uses.  If there are no units
  for your problem, like for a general purpose library, you should still include
the heading, with the content ``not applicable'' (or similar).}

\subsection{Table of Symbols}

The table that follows summarizes the symbols used in this document along with
their units.  The choice of symbols was made to be consistent with the heat
transfer literature and with existing documentation for solar water heating
systems.  The symbols are listed in alphabetical order.

\renewcommand{\arraystretch}{1.2}
%\noindent \begin{tabularx}{1.0\textwidth}{l l X}
\noindent \begin{longtable*}{l l p{12cm}} \toprule
\textbf{symbol} & \textbf{unit} & \textbf{description}\\
\midrule 
$A_C$ & \si[per-mode=symbol] {\square\metre} & coil surface area
\\
$A_\text{in}$ & \si[per-mode=symbol] {\square\metre} & surface area over 
which heat is transferred in
\\ 
\bottomrule
\end{longtable*}
\plt{Use your problems actual symbols.  The si package is a good idea to use for
  units.}
\plt{For the case of a generic numerical library, units will likely not be
  included.  For instance, a linear ODE solver will not know the units of its
  coefficients.}

\subsection{Abbreviations and Acronyms}

\renewcommand{\arraystretch}{1.2}
\begin{tabular}{l l} 
  \toprule		
  \textbf{symbol} & \textbf{description}\\
  \midrule 
  A & Assumption\\
  DD & Data Definition\\
  GD & General Definition\\
  GS & Goal Statement\\
  IM & Instance Model\\
  LC & Likely Change\\
  PS & Physical System Description\\
  R & Requirement\\
  SRS & Software Requirements Specification\\
  \famname{} & \plt{put your famram name here}\\
  T & Theoretical Model\\
  \bottomrule
\end{tabular}\\

\plt{Add any other abbreviations or acronyms that you add.}
\plt{Only include abbreviations and acronyms that are actually used.}

\newpage

\tableofcontents

~\newpage

\pagenumbering{arabic}

\section{Introduction}

\plt{This CA template is based on \citet{Smith2006}.  It
  will get you started, but you will have to make changes.  Any changes to
  section headings should be approved by the instructor, since that implies a
  deviation from the template.  Although the bits shown below do not include
  type information, you may need to add this information for your problem.}

\plt{Feel free to change the appearance of the report by modifying the LaTeX
  commands.}

\subsection{Purpose of Document}

\subsection{Scope of the Family} 

\subsection{Characteristics of Intended Reader} 

\subsection{Organization of Document}

\section{General System Description}

This section identifies the interfaces between the system and its environment,
describes the potential user characteristics and lists the potential system
constraints.

\subsection{Potential System Contexts}

\plt{Your system context will likely include an explicit list of user and system
  responsibilities}

\begin{itemize}
\item User Responsibilities:
\begin{itemize}
\item 
\end{itemize}
\item \famname{} Responsibilities:
\begin{itemize}
\item Detect data type mismatch, such as a string of characters instead of a
  floating point number
\item 
\end{itemize}
\end{itemize}

\subsection{Potential User Characteristics} \label{SecUserCharacteristics}

The end user of \famname{} should have an understanding of undergraduate Level
1 Calculus and Physics.

\subsection{Potential System Constraints}

\plt{You may not have any system constraints.}

\plt{If you need to make design decisions for your family, these decisions will
  be made here as constraints.  For instance, if all inputs will have to use the
same file format, this would be a constraint that would be included here.}

\plt{You should generally limit the number of constraints, to keep the CA
  abstract.}

\section{Commonalities}

\subsection{Background Overview} \label{Sec_Background}

\subsection{Terminology and  Definitions}

This subsection provides a list of terms that are used in the subsequent
sections and their meaning, with the purpose of reducing ambiguity and making it
easier to correctly understand the requirements:

\begin{itemize}

\item 

\end{itemize}

\subsection{Data Definitions} \label{sec_datadef}

This section collects and defines all the data needed to build the instance
models. The dimension of each quantity is also given.  \plt{Modify the examples
  below for your problem, and add additional definitions as appropriate.}

~\newline

\noindent
\begin{minipage}{\textwidth}
\renewcommand*{\arraystretch}{1.5}
\begin{tabular}{| p{\colAwidth} | p{\colBwidth}|}
\hline
\rowcolor[gray]{0.9}
Number& DD\refstepcounter{datadefnum}\thedatadefnum \label{FluxCoil}\\
\hline
Label& \bf Heat flux out of coil\\
\hline
Symbol &$q_C$\\
\hline
% Units& $Mt^{-3}$\\
% \hline
  SI Units & \si{\watt\per\square\metre}\\
  \hline
  Equation&$q_C(t) = h_C (T_C - T_W(t))$, over area $A_C$\\
  \hline
  Description & 
                $T_C$ is the temperature of the coil (\si{\celsius}).  $T_W$ is the temperature of the water (\si{\celsius}).  
                The heat flux out of the coil, $q_C$ (\si{\watt\per\square\metre}), is found by
                assuming that Newton's Law 
                of Cooling applies (\aref{A_Newt_coil}).  This law (\dref{NL}) is used on the surface of
                the coil, which has area $A_C$ (\si{\square\metre}) and heat 
                transfer coefficient $h_C$
                (\si{\watt\per\square\metre\per\celsius}).  This equation
                assumes that the temperature of the coil is constant over time (\aref{A_tcoil}) and that it does not vary along the length
                of the coil (\aref{A_tlcoil}).
  \\
  \hline
  Sources& Citation here\\
  \hline
  Ref.\ By & \iref{ewat}\\
  \hline
\end{tabular}
\end{minipage}\\

\subsection{Goal Statements}

\noindent Given the \plt{inputs}, the goal statements are:

\begin{itemize}

\item[GS\refstepcounter{goalnum}\thegoalnum \label{G_meaningfulLabel}:] \plt{One
    sentence description of the goal.  There may be more than one.  Each Goal
    should have a meaningful label.}

\end{itemize}

\subsection{Theoretical Models} \label{sec_theoretical}

This section focuses on the general equations and laws that \famname{} is based
on.  \plt{Modify the examples below for your problem, and add additional models
  as appropriate.}

~\newline

\noindent
\begin{minipage}{\textwidth}
\renewcommand*{\arraystretch}{1.5}
\begin{tabular}{| p{\colAwidth} | p{\colBwidth}|}
  \hline
  \rowcolor[gray]{0.9}
  Number& T\refstepcounter{theorynum}\thetheorynum \label{T_COE}\\
  \hline
  Label&\bf Conservation of thermal energy\\
  \hline
  Equation&  $-{\bf \nabla \cdot q} + g$ = $\rho C \frac{\partial T}{\partial t}$\\
  \hline
  Description & 
                The above equation gives the conservation of energy for transient heat transfer in a material
                of specific heat capacity $C$ (\si{\joule\per\kilogram\per\celsius}) and density $\rho$ 
                (\si{\kilogram\per\cubic\metre}), where $\bf q$ is the thermal flux vector (\si{\watt\per\square\metre}),
                $g$ is the volumetric heat generation
                (\si{\watt\per\cubic\metre}), $T$ is the temperature
                (\si{\celsius}),  $t$ is time (\si{\second}), and $\nabla$ is
                the gradient operator.  For this equation to apply, other forms
                of energy, such as mechanical energy, are assumed to be negligible in the
                system (\aref{A_OnlyThermalEnergy}).  In general, the material properties ($\rho$ and $C$) depend on temperature.\\
  \hline
  Source &
           \url{http://www.efunda.com/formulae/heat_transfer/conduction/overview_cond.cfm}\\
  % The above web link should be replaced with a proper citation to a publication
  \hline
  Ref.\ By & \dref{ROCT}\\
  \hline
\end{tabular}
\end{minipage}\\

~\newline

\plt{In a CA, the TMs often do not need to be refined.  However, this is not a
  rule.  In some cases, it may make sense to introduce an IM, or possibly even a
  GD in between the TM and the IM.}

\section{Variabilities}

\plt{The variabilities are summarized in the following subsections.  They may
  each be summarized separately, like in \citet{SmithMcCutchanAndCarette2017}, or
  in a table, as in \citet{Smith2006}.}

\plt{For each variability, a description should be given, along with the
  parameters of variation and the binding time.  The parameters of variation
  give the type that defines possible values.  The binding time is when the
  variability is set.  The possible values are specification time (scope time),
  build time and run time.}

\subsection{Assumptions}

\begin{itemize}

\item[A\refstepcounter{assumpnum}\theassumpnum \label{A_meaningfulLabel}:]
  \plt{Short description of each assumption.  Each assumption
    should have a meaningful label.  Use cross-references to identify the
    appropriate traceability to T, GD, DD etc., using commands like dref, ddref etc.}

\end{itemize}

\plt{Input assumptions will be appropriate for many problems.  Some input will
  have simplifying constraints, and other inputs will not.}

\subsection{Calculation} \label{sec_Calculation}

\plt{The calculation variabilities should be as abstract as possible.  If there
  are variabilities that are related to imposed design decisions, the system
  constraints section should be referenced for the relevant constraint.  Design
  constraint related variabilities should be listed separately.}

\plt{Variabilities related to data structure choices would go in this section.
  However, these variabilities are related to design, so they should be
  separated from the more abstract variabilities.}

\plt{Algorithmic variations would go here as well, but as for data structures,
  they should be separated from the more abstract variabilities.}

\subsection{Output} \label{sec_Output}    

\section{Requirements}

This section provides the functional requirements, the business tasks that the
software is expected to complete, and the nonfunctional requirements, the
qualities that the software is expected to exhibit.

\subsection{Family of Functional Requirements}

\plt{Since the CA will often be applied to a library, the functionality will not
  be a single use case.  Therefore, this section should summarize the family of
  potential requirements.  A good way to provide an overview of the functional
  requirements would be to provide multiple use cases on how the library will be
  employed.}

\noindent \begin{itemize}

\item[R\refstepcounter{reqnum}\thereqnum \label{R_Inputs}:] \plt{Requirements
    for the inputs that are supplied by the user.  This information has to be
    explicit.}

\item[R\refstepcounter{reqnum}\thereqnum \label{R_OutputInputs}:] \plt{It isn't
    always required, but often echoing the inputs as part of the output is a
    good idea.}

\item[R\refstepcounter{reqnum}\thereqnum \label{R_Calculate}:] \plt{Calculation
    related requirements.}

\item[R\refstepcounter{reqnum}\thereqnum \label{R_VerifyOutput}:]
  \plt{Verification related requirements.}

\item[R\refstepcounter{reqnum}\thereqnum \label{R_Output}:] \plt{Output related
    requirements.}

\end{itemize}

\subsection{Nonfunctional Requirements}

\plt{To allow the Non-Functional Requirements (NFRs) to vary between family
  members, try to parameterize them.  The value of the parameter is than a variability.}

\plt{An important variability between family members it the relative importance
  of the NFRs.  \citet{Smith2006} shows how pairwise comparisons can be used to
  rank the importance of NFRs.}

\plt{List your nonfunctional requirements.  You may consider using a fit
  criterion to make them verifiable.}

\section{Likely Changes}    

\noindent \begin{itemize}

\item[LC\refstepcounter{lcnum}\thelcnum\label{LC_meaningfulLabel}:] \plt{If
    there is a ranking of variabilities, or combinations of variabilities, that
    are more likely, this information can be included here.}

\end{itemize}

\section{Traceability Matrices and Graphs}

\plt{You will have to add tables.}

\newpage

\bibliographystyle {plainnat}
\bibliography {../../refs/References}

\newpage

\section{Appendix}

\plt{Your report may require an appendix.  For instance, this is a good point to
show the values of the symbolic parameters introduced in the report.}

\subsection{Symbolic Parameters}

\plt{The definition of the requirements will likely call for SYMBOLIC\_CONSTANTS.
Their values are defined in this section for easy maintenance.}

\noindent \plt{Advice on using the template:
\begin{itemize}
\item Assumptions have to be invoked somewhere
\item ``Referenced by'' implies that there is an explicit reference
\item Think of traceability matrix, list of assumption invocations and list of
  reference by fields as automatically generatable
\item If you say the format of the output (plot, table etc), then your
  requirement could be more abstract
\item For families the notion of binding time should be introduced
\item Think of families as a library, not as a single program
\end{itemize}
}

\subsection{First Stage of Implementation}

\plt{In this section specify the family member, or sub-family, that you will be
  implementing.  You should specify the value for all of your
  variabilities, along with the binding time.  A tabular representation will
  probably be the easiest way to convey this information.}

\newpage{}

\subsection{Reflection}

The information in this section will be used to evaluate the team members on the
graduate attribute of Lifelong Learning.  Please answer the following questions:

\begin{enumerate}
  \item What knowledge and skills will the team collectively need to acquire to
  successfully complete this capstone project?  Examples of possible knowledge
  to acquire include domain specific knowledge from the domain of your
  application, or software engineering knowledge, mechatronics knowledge or
  computer science knowledge.  Skills may be related to technology, or writing,
  or presentation, or team management, etc.  You should look to identify at
  least one item for each team member.
  \item For each of the knowledge areas and skills identified in the previous
  question, what are at least two approaches to acquiring the knowledge or
  mastering the skill?  Of the identified approaches, which will each team
  member pursue, and why did they make this choice?
\end{enumerate}

\end{document}